\chapter{پیاده‌سازی، آزمون و ارزیابی}
\section{مقدمه}

در این گزارش، مساله تولید خودکار شرح بر تصاویر را معرفی نموده و سیر تکاملی پژوهش‌های انجام شده در این حوزه را به طور اجمالی مرور کرده و روش‌های مختلفی که در این حوزه مورد استفاده قرار گرفتند، رویکردها، نقاط ضعف و قوت هر یک از این روش‌ها و روند فعلی پژوهش‌ها در این حوزه را بررسی نمودیم. در ادامه، چارچوب کاری رمزگذار-رمزگشا را، که در حال حاضر عمده پژوهش‌گران را به خود جلب کرده است، در حالت استاندارد مطالعه نموده و نقاط ضعف آن را بیان کردیم. 
\\
در فصل گذشته، ایده اصلی پژوهش را، جایگزینی بردار جاسازی کلمات به جای بردار تک‌فعال، مورد بررسی قرار دادیم و ساختار پیشنهادی خود برای بهبود چارچوب کاری رمزگذار-رمزگشا و جنبه‌های مختلف آموزش و تست این ساختار را بیان نمودیم.
\\
در این فصل از گزارش، ابتدا کلیاتی در رابطه با پیاده‌سازی پروژه را بیان خواهیم نمود. شایان ذکر است کد اصلی پروژه به همراه توضیحات جزئی در رابطه با آن، در پیوست اول این گزارش آورده شده است. در ادامه این بخش از گزارش، معیارهای ارزیابی مطرح در حوزه تولید خودکار شرح بر تصاویر را بررسی نموده، نگاهی بر مجموعه‌داده مورد استفاده انداخته و عمل‌کرد ساختار پیشنهادی خود را با روش‌های مشابه دیگر مقایسه می‌نماییم.
\section{پیاده‌سازی}
سامانه تولید خودکار شرح بر تصاویر که در این گزارش، جنبه‌های مختلف آن را مورد بررسی قرار داده‌ایم، به زبان پایتون و با استفاده از نسخه 3.5 آن پیاده‌سازی شده است. در پیاده‌سازی این پروژه، از چارچوب‌ کاری تنسورفلو\enfootnote{Tensorflow} و از کتابخانه جنسیم\enfootnote{Gensim} به منظور بهره‌گیری از ماژول جاسازی کلمات، استفاده شده است. در این بخش از گزارش به معرفی چارچوب کاری تنسورفلو می‌پردازیم. کد اصلی پروژه به همراه توضیحات آن در پیوست اول این گزارش ضمیمه شده است.
\subsection{چارچوب کاری تنسورفلو}
\begin{figure}[H]
	\centering
	\includegraphics[scale=0.2]{Imgs/tensor.png}
\end{figure}

چارچوب کاری تنسورفلو، یکی از جدیدترین و پرکاربردترین چارچوب‌های کاری در حوزه یادگیری ماشین، در زمان نگارش این گزارش، به شمار می‌رود. هسته این چارچوب کاری به زبان \lr{C++} و با استفاده از پلتفرم محاسبات موازی \lr{CUDA} پیاده‌سازی شده است. این چارچوب کاری، یک واسط به زبان پایتون، توسعه داده است که استفاده از آن را در زبان پایتون به سهولت امکان‌پذیر می‌سازد. شایان ذکر است این چارچوب کاری توسط تیم \lr{Google Brain} در حال توسعه است و پشتیبانی می‌شود.
\\
ایدئولوژی اصلی در این چارچوب کاری، بیان محاسبات در قالب گراف‌ است. برای استفاده از این چارچوب کاری، کافیست گراف محاسبات مورد نظر، در این چارچوب طراحی و پیاده‌سازی شود. هر نود این گراف، یک واحد محاسباتی را مشخص می‌کند. با این رویکرد می‌توان هر شبکه عصبی پیچیده‌ای را به راحتی پیاده‌سازی نمود. شکل \ref{fig:tensor} یک مثال ساده از چنین گرافی را برای یک شبکه پیش‌رو ساده نمایش می‌دهد که نودهای دایره‌ای در آن، تنسورها را نمایش داده و نودهای مستطیلی، نمایان‌گر گره‌های محاسباتی هستند.
\begin{figure}[h]
	\centering
	\includegraphics[scale=0.4]{Imgs/tensorsamp.png}
	\caption{یک نمونه از گراف‌های محاسباتی در چارچوب کاری تنسورفلو}
	\label{fig:tensor}
\end{figure}

داده‌های ورودی، ثوابت و متغیر‌های قابل آموزش، همگی در قالب تنسور در این چارچوب کاری قابل تعریف هستند. روند حرکت یک تنسور در گراف طراحی شده و اعمال محاسبات هر نود از گراف روی آن، فرایند اجرای شبکه را تشکیل می‌دهد. به همین دلیل، این چارچوب کاری را تنسورفلو نامیده‌اند. 
این چارچوب کاری را با نمونه‌های مشابه خود به طور اجمالی مقایسه می‌نماید.


\section{معیار \lr{BLEU} \cite{papineni2002bleu}}
معیار
\lr{BLEU}\enfootnote{Bidirectional Evaluation Understudy}
، یکی از معیارهای ارزیابی مدل‌های ترجمه ماشینی است که در حوزه تولید خودکار شرح بر تصاویر نیز مورد استفاده قرار می‌گیرد. در حوزه ترجمه ماشینی، ترجمه‌های مختلفی از یک جمله در زبان مبدا، می‌توان در زبان مقصد ارائه داد. برای تشخیص بهترین ترجمه بین ترجمه‌های کاندید برای یک جمله در زبان مبدا، می‌توان از این معیار استفاده نمود.
\\
نکاتی که در تخمین این معیار موثر هستند به شرح زیر می‌باشند:
\begin{enumerate}
	\item تعداد \lr{n-gram}های اشتراکی با ترجمه‌های مرجع
	\item طول نامتعارف ترجمه، امتیاز منفی برای ترجمه‌های خیلی بلند یا خیلی کوتاه
	\item امتیاز منفی برای تکرار بیش از حد کلمات
	\item امتیاز منفی برای تکرار کلمات هم‌معنی بیش از حد متعارف
\end{enumerate}


\section{معیار \lr{METEOR}\cite{lavie2007meteor}}
این معیار نیز یکی از معیارهایی است که در حوزه ترجمه ماشینی مورد استفاده قرار می‌گیرد. هدف این معیار، اندازه‌گیری میزان انطباق ترجمه تولید شده توسط ماشین با ترجمه‌های مرجع به صورت کلمه به کلمه است. در صورتی‌که بیش از یک ترجمه مرجع وجود داشته باشد، این معیار با تمام ترجمه‌های مرجع به طور مستقل اندازه‌گیری شده و بیشترین مقدار آن، به عنوان امتیاز ترجمه در نظر گرفته می‌شود.
\\
انطباق کلمه‌ به کلمه کلمات دو ترجمه، با در نظر گرفتن یک هم‌ترازسازی کلمات انجام می‌شود. این هم‌ترازسازی، یک نگاشت بین کلمات ترجمه تولیدشده توسط ماشین به حداکثر یکی از کلمات موجود در ترجمه مرجع است که برای هر انطباق میزان انطباق به صورت یک متغیر سه مقداره ذخیره می‌شود. مقادیر این متغیر می‌تواند یکی از موارد زیر باشد:
\begin{enumerate}
	\item انطباق دقیق
	\item انطباق ریشه کلمات
	\item انطباق معنی کلمات
\end{enumerate}

پس از یافتن تمام انطباق‌های مذکور بین دو ترجمه، بلندترین زیردنباله مشترک از این انطباق‌ها بین دو ترجمه انتخاب می‌شود. با توجه به تعداد کلمات موجود در ترجمه‌ تولید شده و ترجمه مرجع، تعداد انطباق‌های یافت شده در بلندترین زیردنباله مشترک و تعداد کلمات موجود بدون انطباق در ترجمه مرجع، دقت\enfootnote{Precision} و به‌خاطرسپاری\enfootnote{Recall} انطباق‌ها محاسبه شده و بین آن‌ها میانگین هارمونیک محاسبه می‌شود. 
\\
در این معیار، پنالتی‌هایی برای خطای موجود در هم‌ترازسازی در نظر گرفته می‌شود. اگر دو ترجمه دقیقا مطابق یک‌دیگر باشند، بیشترین امتیاز در این معیار را کسب می‌کنند. اگر با حذف بزرگترین زیردنباله مشترک از انطباق‌ها، ترجمه ماشینی تولیدشده به قطعات ریز دیگری تبدیل شود، هرچه تعداد این قطعات بیشتر باشد، امتیاز بیشتری از نتیجه نهایی کسر می‌شود. 

\section{معیار \lr{ROUGE-L}\cite{lin2004rouge}}
معیار \lr{ROUGE} \enfootnote{Recall Oriented Understudy Gisting Evaluation}، یکی از معیارهای مورد استفاده در خلاصه‌سازی متن است. این معیار، کیفیت یک خلاصه کاندید ارائه شده برای یک متن را در قبال خلاصه‌های مرجع دیگر می‌سنجد. شاخه‌های متنوعی از این معیار، در بین پژوهش‌های مربوط به خلاصه‌سازی خودکار متون ارائه شده است که در حوزه تولید خودکار شرح بر تصاویر، یکی از این شاخه‌ها مورد توجه قرار می‌گیرد. در این قسمت ما معیار \lr{ROUGE-L} را که در بین پژوهش‌های مرتبط مورد استفاده قرار می‌گیرد، بررسی می‌نماییم.
\\
این معیار، بلندترین زیردنباله مشترک بین خلاصه تولیدشده توسط ماشین و خلاصه‌های مرجع را محاسبه کرده و بر اساس این بلندترین زیردنباله مشترک، به خلاصه تولید شده امتیاز می‌دهد. 

\section{معیار \lr{CIDEr}\cite{vedantam2015cider}}
معیار 
\lr{CIDEr}\enfootnote{Consensus-Based Image Description Evaluation}،
برخلاف معیارهای دیگر، در بین پژوهش‌گران حوزه تولید خودکار شرح بر تصاویر ارائه شده است. این معیار در سال 2015 توسط ودانتام و همکارانش ارائه شد. هدف اصلی این معیار این است که توافق شرح‌های مرجع تولید شده توسط انسان را یافته و سپس میزان انطباق شرح تولید شده خودکار با این توافق را اندازه‌گیری نماید.
\\
در این معیار، ابتدا با اندازه‌گیری \lr{TF-IDF} برای هر کلمه، میزان اهمیت کلمات و معانی آن‌ها در شرح تولید شده، محاسبه می‌شود. این میزان اهمیت به عنوان وزن کلمه در محاسبات بعدی مورد استفاده قرار می‌گیرد. توصیفات تولید شده به طور خودکار و توصیفات مرجع،‌ به صورت مجموعه‌ای از عبارات \lr{n-gram} توصیف می‌شوند. این توصیف از عبارات شامل ۱ تا ۴ کلمه پشت‌سرهم را شامل می‌شوند. سپس میزان انطباق این توصیفات با یک‌دیگر بر اساس وزن کلمات و عبارات، محاسبه می‌شود. در نهایت با استفاده از یک معیار شباهت کسینوسی، میزان شباهت توصیفات تولید شده محاسبه شده و به عنوان امتیاز هر یک از دسته‌های ۱ تا ۴ تایی، گزارش می‌شوند.
\section{آزمایشات}
در این قسمت ابتدا به معرفی مجموعه‌داده مورد استفاده پرداخته و سپس آزمایشات انجام شده و نتایج بدست آمده را گزارش می‌نماییم. همین‌طور نمونه‌هایی از خروجی‌های صحیح و غلط مدل ارائه شده را نمایش می‌دهیم.
\\
در تمامی آزمایشات انجام شده در این بخش، یک مدل پشته‌ای 20 لایه، با بردار حالت 1024 بعدی توسط مجموعه‌داده آموزشی، آموزش داده شده است. نرخ یادگیری در تمام این موارد 0.001 و احتمال حذف نود برابر 0.5 در نظر گرفته شده است. فضای جاسازی کلمات، یک فضای 1024 بعدی در نظر گرفته شده است و قبل از شروع فرایند آموزش شبکه، مدل جاسازی کلمات روی تمام جملات موجود در مجموعه‌داده آموزشی، آموزش داده می‌شود.

\subsection{معرفی مجموعه‌داده}
مجموعه‌داده مورد استفاده، مجموعه‌داده \lr{MS‌‌-COCO} است که توسط مایکروسافت فراهم شده است. پژوهش \cite{lin2014microsoft}، اطلاعات کاملی در رابطه با این مجموعه‌داده فراهم کرده است. 
\\
در این مجموعه‌داده، حدود 83000 تصویر از موضوعات مختلف گردآوری شده به همراه مجموعا حدود 414000 شرح متناظر با تصاویر، در مجموعه‌داده آموزشی و در هر کدام از مجموعه‌داده‌های اعتبارسنجی\enfootnote{Validation} و آزمون، حدود 41000 تصویر با حدود ۲۰۰۰۰۰ شرح متناظر با تصاویر وجود دارد. هر تصویر به طور میانگین دارای 5 شرح مختلف است که توسط انسان تولید شده است. شایان ذکر است تصاویر موجود، در ابعاد مختلف گرد‌آوری شده‌اند. 
\subsection{نمونه‌هایی از خروجی‌ها}
در ادامه، نمونه‌هایی از تصاویر ورودی را به همراه شرح تولید‌شده توسط مدل ارائه شده در این پژوهش مشاهده می‌نمایید. 
\subsubsection{خروجی‌های صحیح}
در جدول \ref{tbl:corr}، نمونه‌هایی از تصاویری موجود در مجموعه‌داده که شرح تولید‌شده توسط مدل پیشنهادی صحیح است را مشاهده می‌نمایید. در این جدول، شرح تولید شده به طور خودکار برای هر تصویر، به همراه مجموعه‌ شرح‌های مرجع برای آن تصویر قابل مشاهده است.
\begin{table}[h]
	\centering
	\caption{نمونه‌هایی از خروجی‌های صحیح مدل ارائه شده در پژوهش}
	\label{tbl:corr}
	\begin{tabular}{|c|c|c|}
		\hline
		تصویر نمونه & شرح تولید شده & شرح مرجع
		\\
		\hline
		\includegraphics[scale=0.1]{Imgs/cor1.jpg}
		& 
		\lr{Giraffe standing in a tree filled area}
		& 
		\begin{tabular}{c}
			\lr{A giraffe standing next to a forest filled with trees.}\\
			\lr{A giraffe eating food from the top of the tree.}\\
			\lr{A giraffe standing up nearby a tree.}\\
			\lr{A giraffe mother with its baby in the forest.}\\
			\lr{Two giraffes standing in a tree filled area.}\\
		\end{tabular}
		\\
		\hline
		\includegraphics[scale=0.1]{Imgs/cor2.jpg}
		& 
		\lr{Man on a surf board in the ocean}
		& 
		\begin{tabular}{c}
			\lr{A‌ man laying on a surfboard in the water.}\\
			\lr{A man lying on a surfboard in some small waves in water.}\\
			\lr{A young man paddles a surfboard in the ocean.}\\
			\lr{A giraffe mother with its baby in the forest.}\\
			\lr{The person is riding his surfboard out in the ocean.}\\
		\end{tabular}
		\\
		\hline
		\includegraphics[scale=0.1]{Imgs/cor3.jpg}
		& 
		\lr{Plane is on the ground on the tarmac}
		& 
		\begin{tabular}{c}
			\lr{A modern jet airliner on a snow edged runway.}\\
			\lr{The airplane is on the ground on the runway.}\\
			\lr{The front view of a white jet on an airports runway.}\\
			\lr{A‌ large air plane on an airport runway.}\\
			\lr{A plane that has jet landed at the airport.}\\
		\end{tabular}
		\\
		\hline
	\end{tabular}
\end{table}
\subsubsection{خروجی‌های غلط}
جدول \ref{tbl:incorr}، نمونه‌هایی از تصاویر را که شرح تولید‌شده برای آن‌ها غلط بوده، نمایش می‌دهد. 
\begin{table}[h]
	\centering
	\caption{نمونه‌هایی از خروجی‌های غلط مدل ارائه شده در پژوهش}
	\label{tbl:incorr}
	\begin{tabular}{|c|p{2.5cm}|c|}
		\hline
		تصویر نمونه & شرح تولید شده & شرح مرجع
		\\
		\hline
		\includegraphics[scale=0.1]{Imgs/incor1.jpg}
		& 
		\lr{Old man in a coat has a giraffis look on his face}
		& 
		\begin{tabular}{c}
			\lr{A man stands with a frown on his face.}\\
			\lr{A old man in coat and tie walking down a busy street.}\\
			\lr{An old man in a business suit has a thoughtful face.}\\
			\lr{An older gentleman wearing a suit has a grumpy look on his face.}\\
			\lr{An old man wearing glasses who doesn't look very happy.}\\
		\end{tabular}
		\\
		\hline
		\includegraphics[scale=0.1]{Imgs/incor2.jpg}
		& 
		\lr{Train of an empty intersection with traffic lights}
		& 
		\begin{tabular}{c}
			\lr{Two traffic lights are posted near the street intersection.}\\
			\lr{An empty street with some stop lights in a little island.}\\
			\lr{Topside view of an empty intersection with traffic lights.}\\
			\lr{A street intersection with some traffic lights on the side walk.}\\
			\lr{Some stop signs that are at an intersection.}\\
		\end{tabular}
		\\
		\hline
		\includegraphics[scale=0.1]{Imgs/incor3.jpg}
		& 
		\lr{small black and dog with a frisbee by its feet}
		& 
		\begin{tabular}{c}
			\lr{A small dog standing on a wet ground looking up.}\\
			\lr{A small black and white dog standing on a sparse grass looking at ahuman.}\\
			\lr{A small black and white dog standing next to a pink and black frisbee.}\\
			\lr{A dog looking up at a person with a frisbee at their feet.}\\
		\end{tabular}
		\\
		\hline
	\end{tabular}
\end{table}

\subsection{مقایسه با روش‌های مشابه}
\subsubsection{تعداد پارامترها}
دیکشنری کلمات ایجاد شده روی مجموعه‌داده آموزشی مورد استفاده، بالغ بر 23000 کلمه را شامل می‌شود. استفاده از بردار تک‌فعال به عنوان بردار مطلوب و پیش‌بینی توزیع احتمال کلمات در هر مرحله، با ساختار مشخص‌شده شبکه، منجر به ایجاد حدود 400 میلیون پارامتر قابل آموزش می‌شود. با جای‌گزینی مدل جاسازی کلمات در ابعاد 1024‌بعدی، ساختار یکسان، برای رگرسیون جاسازی کلمات، تنها 86 میلیون پارامتر قابل آموزش خواهد داشت. به عبارت دیگر، تعداد پارامترهای قابل آموزش شبکه، در حالت استفاده از جاسازی کلمات، حدود 21.5 درصد تعداد پارامترهای مورد نیاز برای بردار تک‌فعال است.
\\
جدول \ref{tbl:par1}، روند رشد تعداد وزن‌های شبکه را به ازای 

علاوه بر مورد فوق، شبکه در حالت استفاده از جاسازی کلمات، برای هم‌گرایی نیاز به حدود 20 تکرار دارد. این درحالی است که همین ساختار برای استفاده از بردار تک‌فعال، در حدود 110 تکرار به هم‌گرایی می‌رسد و این نشان‌گر این نکته است که کاهش ابعاد خروجی و حذف بردارهای تنک باعث سهولت یادگیری شبکه و افزایش سرعت هم‌گرایی در شبکه می‌شود.

\subsubsection{معیار \lr{BLEU}}

\section{بحث درباره نتایج و عملکرد شبکه}

\section{جمع‌بندی}













