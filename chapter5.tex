\chapter{جمع‌بندي، نتيجه‌گيري و پیشنهادات}
%%%%%%%%%%%%%%%%%%%%%%%%%%%%%%%%%%%%%%%%%%%
تولید و ذخیره‌سازی روزافزون تصاویر، سهولت در استفاده از دوربین‌های تصویر‌برداری و گوشی‌های موبایل، دسترسی آسان به اینترنت در تمام نقاط شهر و افزایش تعداد شبکه‌های اجتماعی و نرم‌افزارهای موبایل، باعث افزایش نیاز کاربران به سامانه‌های هوشمند مدیریت تصاویر شده است. سامانه‌هایی که علاوه بر مدیریت ذخیره و بازیابی تصاویر، قدرت دسته‌بندی خودکار، جستجوی محتوایی، درک و توصیف تصاویر از هر موضوعی باشند.
\\
ارائه مدل‌های هوشمند که بتوانند به طور خودکار برای هر تصویری، توصیف متناظر در قالب جملات زبان طبیعی تولید کنند، از جمله مهم‌ترین اقدامات در راستای رسیدن به سامانه مدیریت تصاویر به شمار می‌رود. 
\\
روند پژوهش در مساله تولید خودکار شرح بر تصاویر، به سال‌های 1970 تا 1980 بر می‌گردد. در این سال‌ها اولین پژوهش‌ها با موضوع شناخت فرایند توصیف تصویر توسط مغز انسان انجام شد. پژوهش‌گران در این پژوهش‌ها، با بررسی افرادی که در یک فضای کنترل‌شده، تصاویر متفاوتی را توصیف می‌کنند، سعی در شناخت ویژگی‌های فرایند توصیف تصویر توسط مغز انسان داشتند. فرایند توصیف تصویر کاز دو مرحله استخراج اطلاعات تصویر و بیان اطلاعات استخراج شده در قالب جملات تشکیل شده است. طی بررسی‌های انجام شده، مشخص شد مغز انسان در کمتر از حدود ۲۰۰ تا ۵۰۰ میلی‌ثانیه، قادر است تمام اطلاعات مورد نیاز برای توصیف تصویر را استخراج نماید. 
\\
با طرح مساله و چالش جدید، توجه پژوهش‌گران به حل این مساله، جلب شد. در سال‌های قبل از 2000، عموم چالش‌های برجسته در این حوزه، مربوط به بخش درک صحنه و استخراج اطلاعات موجود در تصویر بود. در این سال‌ها عموما،‌ پژوهش‌گران با محدود کردن موضوع تصاویر مورد بررسی و با استفاده از ویژگی‌های تصویر مانند لبه‌ها،‌ توصیف‌کننده‌های مختلف، بافت و موارد مشابه دیگر، سعی در استخراج اطلاعات تعریف شده داشتند.
\\
در سال‌های 2000 تا 2014، با وارد شدن روش‌های گرافی احتمالی، به حوزه تولید خودکار شرح بر تصاویر، به مرور زمان محدودیت‌های گذشته برطرف شد و فرایند استخراج اطلاعات از تصاویر به خوبی و بدون نیاز به محدودکردن موضوع تصاویر، انجام می‌شد. مدل‌های گرافی احتمالی زیادی در این مسیر به کار گرفته شد. از جمله این مدل‌ها می‌توان به مدل میدان تصادفی مارکف و میدان تصادفی شرطی اشاره کرد. همین‌طور پژوهش‌گرانی هم وجود داشتند که برای حل این چالش، مدل جدیدی را طراحی و تولید نمایند که نمونه‌هایی از آن‌ها را در بخش مرور مطالعات پیشین بررسی نمودیم.
\\
در سال‌های بعد از 2007، می‌توان گفت توجه پژوهش‌گران بیشتر به سمت مدل‌های تولید جمله جلب شد و چالش‌های موجود دراین حوزه که اغلب بدون راه‌حل  بودند یا با راه‌حل‌های ابتدایی حل می‌شدند، بیش از پیش مورد استقبال پژوهش‌گران قرار گرفتند. مدل‌های مختلفی برای تولید جمله به کار گرفته شد. از جمله این مدل‌ها می‌توان به روش‌های موجود در حوزه تولید زبان طبیعی، بازیابی شبیه‌ترین جمله موجود در مجموعه‌داده و استفاده از کلیشه زبانی،‌ اشاره کرد. اما هیچ‌یک از این روش‌ها، نتوانستند تمام معضلات را حل‌نمایند.
\\
اما با حل مشکل ناپایداری آموزش شبکه‌های عصبی بازگشتی در سال 2011 توسط هینتون، فصل جدیدی در حوزه تولید جمله در این مساله شروع شد. شبکه‌های عصبی بازگشتی، ابزارهای قدرتمندی در کاربرد پیش‌بینی دنباله‌های زمانی و تولید جمله به شمار می‌روند. قابلیت‌های بالای این مدل‌ها، پژوهش‌گران را برآن داشت که تمامی روش‌های گذشته را کنار گذاشته و تماما از شبکه‌های عصبی بازگشتی برای تولید جمله استفاده نمایند.
\\
استفاده از شبکه‌های عصبی بازگشتی، ذهن اغلب پژوهش‌گران را به سمت استفاده از شبکه‌های کانولوشنی عمیق در مرحله استخراج اطلاعات از تصاویر می‌کشاند. شبکه‌های عصبی کانولوشنی عمیق، در استخراج ویژگی‌های بسیار خوب از تصاویر، قدرت بالایی دارند. از حدود سال 2014 به بعد و با جابجایی مدل‌های گرافی احتمالی با شبکه‌های عصبی کانولوشنی عمیق، صفر تا صد فرایند تولید خودکار شرح بر تصاویر، با استفاده از شبکه‌های عصبی و یادگیری عمیق انجام می‌شد.
\\
با گرایش تمامی پژوهش‌ها در این حوزه به یادگیری عمیق، عمل‌کرد مدل‌های ارائه شده، دچار یک جهش اساسی در این زمینه شد. همین‌طور در سال 2015،  چارچوب کاری رمزگذار-رمزگشا، یکی از مشهورترین چارچوب‌های کاری یادگیری عمیق، که پیش‌تر کاربردهای بسیار زیادی در حوزه ترجمه ماشینی یافته  بود، پا به عرصه پژوهش‌های مرتبط با تولید خودکار شرح بر تصاویر نهاد. از آن پس، این چارچوب کاری تبدیل به جزو لاینفکی از پژوهش‌های حوزه تولید خودکار شرح بر تصاویر شده است.
\\
ورود چارچوب کاری رمزگذار-رمزگشا از حوزه ترجمه ماشینی به حوزه تولید خودکار شرح بر تصاویر، باب انتقال روش‌های دیگر را بین این دو حوزه، بیش از پیش گشود. در سال‌های اخیر، روش‌های مبتنی بر توجه که در حوزه ترجمه ماشینی، عمل‌کرد بسیار خوبی از خود نشان دادند، به حوزه تولید خودکار شرح بر تصاویر وارد شدند. در حال حاضر، بیش‌ترین پژوهش‌های انجام شده در حوزه تولید خودکار شرح بر تصاویر،‌ مبتنی بر روش‌های توجه بصری هستند.
\\
چارچوب کاری رمزگذار-رمزگشا در حوزه تولید خودکار شرح بر تصاویر در تمامی پژوهش‌ها، تقریبا به طور مشابه مورد استفاده قرار می‌گیرد. رمزگذار مورد استفاده در این چارچوب کاری، یک شبکه عصبی کانولوشنی عمیق است که به‌ ازای هر تصویر ورودی،‌ یک یا چند بردار ویژگی تولید می‌نماید. در پژوهش‌های مختلف، از شبکه‌های عصبی کانولوشنی مختلفی به این منظور استفاده می‌شود. عموما استفاده از این شبکه‌ها به طور از پیش آموزش دیده، اتفاق می‌افتد.
\\
وظیفه واحد رمزگشا در چارچوب کاری رمزگذار-رمزگشا این است که با دریافت  بردار ویژگی استخراج شده یا بردارهای حاشیه‌نویسی تصویر، جملات توصیف‌کننده مربوط به تصویر را تولید نماید. هر جمله را می‌توان به عنوان دنباله‌ای با طول متغیر از کلمات در نظر گرفت که هر کلمه در آن، با یک بردار ویژگی توصیف شده است. در تمام پژوهش‌های انجام شده در این حوزه،‌ رمزگشا در هر مرحله،‌ سعی در پیش‌بینی توزیع احتمال شرطی کلمه بعدی به شرط داشتن کلمات تولید شده قبلی و بردار ویژگی تصویر می‌نماید. 
\\
به منظور آموزش شبکه رمزگشا به نحوی که قادر به تولید توزیع احتمال شرطی کلمات بعدی به شرط داشتن کلمات قبلی و بردار ویژگی تصویر باشد، باید مقادیر مطلوب در هر مرحله، به شکل یک توزیع احتمال روی تمام کلمات، به شبکه داده شود. برای این‌کار، در هر مرحله از یک بردار تک‌فعال به عنوان بردار مطلوب استفاده می‌‌شود. 
\\
استفاده از بردار تک‌فعال به عنوان بردار مطلوب در هر مرحله مشکلاتی را به همراه دارد که از آن‌ جمله می‌توان به نکات زیر اشاره کرد:
\begin{enumerate}
	\item عدم استفاده از اطلاعات معنایی کلمات، استفاده از کلمات مترادف و تولید جملات متنوع‌تر
	\item تنک‌شدن مقدار مطلوب شبکه و کاهش قدرت شبکه در یادگیری پارامترها
	\item افزایش چشم‌گیر تعداد پارامترهای قابل یادگیری شبکه به دلیل بالابودن ابعاد خروجی در تمام مراحل
\end{enumerate}
ایده اصلی ارائه شده در این پژوهش، استفاده از بردار جاسازی کلمات به جای بردار تک‌فعال به عنوان مقدار مطلوب در هر مرحله است. بر همین اساس، مدل رمزگشا به نحوی تغییر می‌کند که رمزگشا به جای پیش‌بینی توزیع احتمال شرطی کلمات، بردار جاسازی کلمه بعدی را در هر مرحله تولید نماید. به عبارت دیگر، مساله پیش‌بینی توزیع احتمال شرطی در شبکه رمزگشا، با تغییر مذکور، به یک مساله رگرسیون تغییر پیدا می‌کند.
\\
بردار جاسازی مورد استفاده در این پژوهش، با استفاده از مدل \lr{Skip-Gram} تولید می‌شود. ویژگی برجسته این مدل در این است که بردار جاسازی کلمات به نحوی تولید می‌شود که با داشتن آن بتوان بردار جاسازی کلمات بعدی را به راحتی دسته‌بندی نمود. این ویژگی، علاوه بر این که مدل رمزگشا را قادر به استفاده از اطلاعات معنایی کلمات می‌کند، یادگیری پیش‌بینی بردارهای جاسازی بعدی در هر مرحله را سهولت می‌بخشد.
\\

